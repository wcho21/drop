\documentclass{cmn}
\usepackage{ts-arrow}
\usetikzlibrary{calc}

\newcommand\tokenfbox[1]{\fbox{\texttt{#1}\vphantom{\strut}}}
\def\arrowWidth{36mm}

\begin{document}
  \begin{tikzpicture}[level distance=14mm]
    \def\plustoken{\tokenfbox{+}}
    \def\onetoken{\tokenfbox{1}}
    \def\twotoken{\tokenfbox{2}}

    \node (A) at (0*\arrowWidth,0) {\texttt{1+2}};
    \node (B) at (1*\arrowWidth,0) {\onetoken\ \plustoken\ \twotoken};
    \coordinate (C) at (2.25*\arrowWidth,0);
    \coordinate (CLeft) at (2*\arrowWidth,0);
    \coordinate (CRight) at (2.5*\arrowWidth,0);
    \coordinate[yshift=7mm] (CRoot) at (C);
    \coordinate[yshift=-7mm] (CChild) at (C);
    \node at (CRoot) {\plustoken}
      child {node {\onetoken}}
      child {node {\twotoken}};
    \node (D) at (3.25*\arrowWidth,0) {3};

    \draw[StlthBase] (A) -> node[above=-0.5mm] {\small tokenize\strut} (B);
    \draw[StlthBase] (B) -> node[above=-0.5mm] {\small parse\strut} (CLeft);
    \draw[StlthBase] (CRight) -> node[above=-0.5mm] {\small evaluate\strut} (D);

    \node[yshift=-8mm] at (A) {\small (Source code)};
    \node[yshift=-10mm] at (B) {\small (Tokens)};
    \node[yshift=-16mm] at (C) {\small (Abstract syntax tree)};
    \node[yshift=-8mm] at (D) {\small (Value)};
  \end{tikzpicture}
\end{document}
