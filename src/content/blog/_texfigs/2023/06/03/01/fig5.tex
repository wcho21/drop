\documentclass{cmn}
\usepackage{ts-arrow}

\tikzset{
  arr/.style={StlthRev}
}

\newcommand\scopeXSep{40mm}
\newcommand\stackwidth{9mm}
\newcommand\stackheight{56pt}
\newcommand\arrowwidth{12pt}
\newcommand\arrowheight{14pt}
\newcommand\labelnodewidth{32pt}

\newcommand\drawstack{
  \draw (0,0) -- ++(0,-\stackheight) -- ++(\stackwidth,0) -- ++(0,\stackheight);
}
\newcommand\drawpoparr{
  \draw[arr] (-\arrowwidth/2,\arrowheight) .. controls +(right:6pt) and +(up:8pt) .. ++(\arrowwidth,-\arrowheight);
}
\newcommand\drawpusharr{
  \draw[arr] (\stackwidth-\arrowwidth/2,0pt) .. controls +(up:8pt) and +(left:6pt) .. ++(\arrowwidth,\arrowheight);
}
\newcommand\drawpoplabel[1]{
  \node[text width=\labelnodewidth,align=right] at (-\labelnodewidth,12pt) {#1};
}
\newcommand\drawpushlabel[1]{
  \node[text width=\labelnodewidth,align=left] at (\stackwidth+\labelnodewidth,12pt) {#1};
}

\begin{document}
  \begin{tikzpicture}
    arr/.style={{Stealth[scale=1.2]}-}
  ]
    \begin{scope}
      \drawstack

      \drawpushlabel{\fbox{1}\,32}
    \end{scope}

    \begin{scope}[xshift=\scopeXSep]
      \drawstack

      \drawpoplabel{\fbox{1}}
      \drawpushlabel{\fbox{32}}

      \drawpoparr
      \drawpusharr
    \end{scope}

    \begin{scope}[xshift=2*\scopeXSep]
      \drawstack

      \node at (\stackwidth/2,-44pt) {3};
      \node at (\stackwidth/2,-24pt) {2};

      \drawpoplabel{1}

      \drawpusharr
    \end{scope}

    \begin{scope}[xshift=3*\scopeXSep]
      \drawstack

      \drawpoplabel{1\,\fbox{23}}

      \drawpoparr
    \end{scope}
  \end{tikzpicture}
\end{document}
